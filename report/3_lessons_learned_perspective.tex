\newpage
\section{Lessons Learned Perspective}

%Describe the biggest issues, how you solved them, and which are major lessons learned with regards to:

%  - evolution and refactoring
%  - operation, and
%  - maintenance

%of your _ITU-MiniTwit_ systems. Link back to respective commit messages, issues, tickets, etc. to illustrate these.


%Also reflect and describe what was the "DevOps" style of your work. For example, what did you do differently to previous development projects and how did it work?

\subsection{Evolution and refactoring} \label{evoref}
As mentioned, we initially wanted to use Blazor to develop the application. However, because of the limited time for refactoring, we switched to Razor pages, as we estimated that it required a third of the code. Furthermore, the code base would be much more similar to the project we had to refactor. Blazor is notoriously known for splitting up the model, view, and controller, which is not the case with Razor.

\subsection{Operations}
We discovered the importance of knowledge sharing. We quickly found that when we did not share what we had done to the project since last week, it took time for all members to have an overview of the entire system. To solve this issue, we started using retrospectives as a way of sharing knowledge every Tuesday before the next week's task.

\subsection{Limited work time}
One of the most significant issues the group faced was our limited work time. Since the project was rather extensive, and the amount of new tools that should be incorporated was large, it required much time from the group members. This was an issue since we primarily only had Tuesday as the day when we met physically. We all agreed that we work better when in the room together, and this was not easy to do every time. The result of this issue was that we sometimes had to work until the late evening hours on Tuesdays and work only on essential parts of the application. Still, it was a price that we were willing to pay not to have to work more online than we already did.

\subsection{Major lessons learned}
\begin{enumerate}
    \item We learned a lot of new tools such as Grafana, Prometheus, Vagrant, and Digital Ocean, among others. We also reinforced our learning and learned more about tools like Bash, Docker, and Github.
    \item We learned how to use the DevOps theory in practice on a project, which helps a lot with understanding the concepts more thoroughly.
    \item We learned what it felt like to work with a DevOps approach, which many companies use in real life.
    \item We learned and understood how one should think when working with DevOps since this is a very different way of thinking about the project and the code compared to not using a DevOps approach.
    \item We learned about the value of automating work: E.g., turning the entire application on/off with a single script.
    \item We learned about the practical sides of writing code meant to be published. 
\end{enumerate}

\subsection{The differences with DevOps}
Many previous projects we have worked with on ITU have not used a DevOps approach. This resulted in many of the things we did, the technologies we used, and how we had to think about the project changed substantially. All group members were writing their bachelor thesis while having this course, and none related to DevOps. This gives an obvious comparison between the applied work ethics in the two respective courses.  \\

Usually, we do not prioritize committing and pushing code as soon as it works in small steps. Instead, we usually only push when we have to merge our code with what the others have written. This is in sharp contrast to the DevOps approach we had in this course, where we would always commit, push (and, when the pipeline was set up correctly, release) when we had something that worked. The DevOps approach led to fewer merge conflicts than in our previous projects, resulting in our code being live faster. If one has code locally on their device without it being released, it is considered dead code since it is not used.